%definisco il layout dell'abstract
\def\changemargin#1#2{\list{}{\rightmargin#2\leftmargin#1}\item[]}
\let\endchangemargin=\endlist

%Genero l'ambiente per l'abstract
\newcommand\summaryname{Abstract}
\newenvironment{Abstract}%
{\begin{center}%
\bfseries{\summaryname} \end{center}}

\begin{Abstract}
    \begin{changemargin}{1cm}{1cm}
        Low Power Wide Area Networks (LPWAN) are getting very popular these days in Internet of Things (IoT) applications
        thanks to their capability of both consuming low amounts of power and of covering long distances. This
        technology is widely used in the 4$^{th}$ industrial era for manufacturing, health care, and automation in general.\\
        This thesis has the objective to propose a Media Access Control (MAC) protocol called Bacco. It is based on LoRa
        modulation and has a narrow focus on agricultural applications, where achieving high power efficiency is crucial due
        to the lack of reliable power sources. Another aspect taken into consideration is the cost effectiveness of the
        devices required to develop a functional network.\\
        First, the thesis establishes an introduction of LoRa and LoRaWAN; then the requirements for a MAC protocol
        used in LPWANs will be discussed. After that, there will be a description of the Bacco protocol itself, alongside with some
        example applications of it.

    \end{changemargin}
\end{Abstract}

\newpage

%%definisco il layout dell'abstract
%\def\changemargin#1#2{\list{}{\rightmargin#2\leftmargin#1}\item[]}
%\let\endchangemargin=\endlist

%Genero l'ambiente per l'abstract
\newcommand\nomesommario{Sommario}
\newenvironment{Sommario}%
{\begin{center}%
\bfseries{\nomesommario} \end{center}}

\begin{Sommario}
    \begin{changemargin}{1cm}{1cm}
        Le reti Low Power Area Network (LPWAN) stanno prendendo piede oggigiorno nel mondo dell'Internet of Things
        (IoT) grazie al loro basso consumo energetico e alle ampie distanze che possono coprire. Questa tecnologia è
        un caposaldo dell'industria di quarta generazione, soprattutto negli ambiti di manifattura, assistenza sanitaria e in
        generale dell'automazione.\\
        Questa tesi ha l'obiettivo di proporre un protocollo Media Access Control (MAC), chiamato Bacco. Questo sfrutta la
        modulazione LoRa e si rivolge a applicazioni in ambito agricolo, dove è cruciale raggiungere
        un'alta efficienza energetica data la mancanza di fonti energetiche affidabili. Un altro aspetto
        che viene considerato è il costo dei dispositivi rischiesti per sviluppare una rete funzionale.\\
        Inizialmente la tesi si occuperà di dare una breve introduzione a LoRa e LoRaWAN, per poi discutere i
        requisiti di un protocollo MAC per LPWAN. Successivamente, verrà data una descrizione del funzionamento di
        Bacco, accompagnata da alcuni esempi applicativi.

    \end{changemargin}
\end{Sommario}
